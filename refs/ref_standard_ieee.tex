% source: CIE6010

\documentclass[10pt,conference]{styles/IEEEtran}


\usepackage{epsfig}
\usepackage{amssymb} 
\usepackage[tbtags]{amsmath} 
\usepackage{graphics,eepic,epic}
\usepackage{latexsym}
%\usepackage{spacing}
\usepackage{euscript}
\usepackage{subfigure}
\usepackage{graphics,eepic,epic,psfrag}



\input{styles/gww_defs}
\input{styles/gww_chars}
\input{styles/defs}


\newcommand{\bl}{n}
\newcommand{\code}{\mathcal{C}}
\newcommand{\neighborhood}{\mathcal{N}}
\newcommand{\oddSet}{\mathcal{V}}
\newcommand{\numChks}{m}
\newcommand{\worstInd}{{i^{\ast}}}

\begin{document}

% paper title
\title{ML decoding via mixed-integer adaptive \\ linear programming}


% author names and affiliations
% use a multiple column layout for up to three different
% affiliations
\author{
\authorblockN{Stark C.~Draper}
\authorblockA{Mitsubishi Electric Research Labs \\
Cambridge, MA 02139 USA\\
draper@merl.com}
\and
\authorblockN{Jonathan S.~Yedidia}
\authorblockA{Mitsubishi Electric Research Labs \\
Cambridge, MA 02139 USA\\
yedidia@merl.com}
\and
\authorblockN{Yige Wang}
\authorblockA{Dept.~of EE, Univ.~of Hawaii at Manoa\\
Honolulu, HI 96822 USA\\
yige@spectra.eng.hawaii.edu}
}

\maketitle

\begin{abstract}

Linear programming (LP) decoding was introduced by Feldman et
al. ({\em IEEE Trans.~Inform.~Theory} Mar.~2005) as a novel way to
decode binary low-density parity-check codes.  Taghavi and Siegel
({\em Proc.~ISIT} 2006) describe a computationally simplified decoding
approach they term ``adaptive'' LP decoding.  Adaptive LP decoding
starts with a sub-set of the LP constraints, and iteratively adds
violated constraints until an optimum of the original LP is
found. Usually only a tiny fraction of the original constraints need
to be reinstated, leading to huge efficiency gains compared to
ordinary LP decoding.

Here we describe a modification of the adaptive LP decoder that
results in a maximum likelihood (ML) decoder. Whenever the adaptive LP
decoder returns a pseudo-codeword rather than a codeword, we add an
integer constraint on the least certain symbol of the pseudo-codeword.
For certain codes, and especially in the high-SNR (error floor)
regime, only a few integer constraints are required to force the
resultant mixed-integer LP to the ML solution.  We demonstrate that
our approach can efficiently achieve the optimal ML decoding
performance on a (155,64) LDPC code introduced by Tanner et al.
\end{abstract}

\input{introduction}
\input{background}
\input{mixedInteger}
\input{numerical}
\input{conclusions}
% An example of a floating figure using the graphicx package.
% Note that \label must occur AFTER (or within) \caption.
% For figures, \caption should occur after the \includegraphics.
%
%\begin{figure}
%\centering
%\includegraphics[width=2.5in]{myfigure}
% where an .eps filename suffix will be assumed under latex,
% and a .pdf suffix will be assumed for pdflatex
%\caption{Simulation Results}
%\label{fig_sim}
%\end{figure}


% An example of a double column floating figure using two subfigures.
%(The subfigure.sty package must be loaded for this to work.)
% The subfigure \label commands are set within each subfigure command, the
% \label for the overall fgure must come after \caption.
% \hfil must be used as a separator to get equal spacing
%
%\begin{figure*}
%\centerline{\subfigure[Case I]{\includegraphics[width=2.5in]{subfigcase1}
% where an .eps filename suffix will be assumed under latex,
% and a .pdf suffix will be assumed for pdflatex
%\label{fig_first_case}}
%\hfil
%\subfigure[Case II]{\includegraphics[width=2.5in]{subfigcase2}
% where an .eps filename suffix will be assumed under latex,
% and a .pdf suffix will be assumed for pdflatex
%\label{fig_second_case}}}
%\caption{Simulation results}
%\label{fig_sim}
%\end{figure*}


% An example of a floating table. Note that, for IEEE style tables, the
% \caption command should come BEFORE the table. Table text will default to
% \footnotesize as IEEE normally uses this smaller font for tables.
% The \label must come after \caption as always.
%
%\begin{table}
%% increase table row spacing, adjust to taste
%\renewcommand{\arraystretch}{1.3}
%\caption{An Example of a Table}
%\label{table_example}
%\begin{center}
%% Some packages, such as MDW tools, offer better commands for making tables
%% than the plain LaTeX2e tabular which is used here.
%\begin{tabular}{|c||c|}
%\hline
%One & Two\\
%\hline
%Three & Four\\
%\hline
%\end{tabular}
%\end{center}
%\end{table}


%\section{Conclusion}
%Never conclude a real information theoretic paper.
%If you have to, the conclusion goes here.

% conference papers do not normally have an appendix

% use section* for acknowledgement
%\section*{Acknowledgment}
% optional entry into table of contents (if used)
%\addcontentsline{toc}{section}{Acknowledgment}
%The authors would like to thank various sponsors for supporting 

% trigger a \newpage just before the given reference
% number - used to balance the columns on the last page
% adjust value as needed - may need to be readjusted if
% the document is modified later
%\IEEEtriggeratref{8}
% The "triggered" command can be changed if desired:
%\IEEEtriggercmd{\enlargethispage{-5in}}

% references section
% NOTE: BibTeX documentation can be easily obtained at:
% http://www.ctan.org/tex-archive/biblio/bibtex/contrib/doc/

% can use a bibliography generated by BibTeX as a .bbl file
% standard IEEE bibliography style from:
% http://www.ctan.org/tex-archive/macros/latex/contrib/supported/IEEEtran/bibtex
%\bibliographystyle{IEEEtran.bst}
% argument is your BibTeX string definitions and bibliography database(s)
%\bibliography{IEEEabrv,../bib/paper}
%
% <OR> manually copy in the resultant .bbl file
% set second argument of \begin to the number of references
% (used to reserve space for the reference number labels box)
\bibliography{references}

\bibliographystyle{plain}
%\begin{thebibliography}{1}


%\bibitem{Shannon1948}
%C. E. Shannon, ``A mathematical theory of communication,''
%\emph{Bell Syst.\ Tech.\ J.}, vol.\ 27, pt.~I, pp.~379--423, 1948;
%     pt.~II, pp.~623--656, 1948.


%\end{thebibliography}


\end{document}
